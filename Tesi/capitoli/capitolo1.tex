\chapter{Introduzione}

\label{cap:introduzione}
\section{Il progetto}
Il progetto sviluppato per questo lavoro di tesi ha riguardato l'attivit\`a del web scraping da fonti social network e la sua applicazione nella programmazione di connettori di dati in sistemi di Big Data Analytics. Lo studio di soluzioni valide ha individuato come obiettivi principali i social di Meta Inc., ovvero Facebook ed Instagram, per la loro vasta utenza e conseguente ingente quantit\`a di dati pubblici. 

Nella tesi vengono affrontate tematiche inerenti alla legalit\`a delle azioni di scraping, in quanto ad oggi, l'attivit\`a ricade in  una ``zona grigia'', non essendo previste direttamente norme. I social tramite i loro termini di servizio, vietano qualsiasi tipologia di estrazione dati, anche adottando soluzioni tecniche, per salvaguardare le informazioni pubblicate dai propri utenti.

Oggetto di attenzione all'interno del progetto \`e anche l'impiego dei dati estratti, presentando valide proposte d'utilizzo per scopi di investigazione e Open Source Intelligence.

Lo sforzo tecnico e progettuale \`e consistito principalmente nella creazione di valide metodologie di elusione dei controlli anti-scraping attuati dalle piattaforme. Lo studio del contrasto ha garantito la progettazione di sistemi attuabili in entrambi i connettori dei social target, uniformando la loro gestione e il loro output.


\section{Organizzazione dei contenuti}
Il presente lavoro di tesi \`e cos\`i strutturato: \\
Il Capitolo \ref{chap:stato_arte} presenta lo stato dell'arte rispetto al web scraping in generale e da fonti social network. Vengono inoltre citati gli scenari d'impiego dei dati in ambito investigativo e gli aspetti giuridici dell'attivit\`a. \\
Il Capitolo \ref{chap:analisi_concettuale} individua i requisiti di un progetto di web scraping presentando le tecnologie di base da impiegare, le soluzioni ideali di elusione dei controlli anti-scraping e la gestione dei dati in output.\\
Nel Capitolo \ref{chap:caso_studio} viene presentato il lavoro di studio e sviluppo svolto per il progetto di tesi. In particolare per i singoli connettori di Facebook ed Instagram vengono proposti i punti salienti del progetto e le soluzioni implementative attuate. Inoltre vengono elencate ed esplicate tutte le tecnologie di sviluppo adottate.\\
Il Capitolo \ref{chap:test} presenta i test effettuati sul software, proponendo un confronto sia sulle versioni dei tool open source impiegati che sull'efficienza in termini di dati, tempo e resistenza dei connettori sviluppati.\\
Nel Capitolo \ref{chap:conclusioni} vengono esposte le conclusioni sul progetto sviluppato, presentando eventuali sviluppi futuri. \\
Infine, nell'Appendice, \ref{appendice} sono presentate parti del codice prodotto ed esempi di output dei connettori. \\

