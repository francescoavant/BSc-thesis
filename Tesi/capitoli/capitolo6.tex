\chapter{Conclusioni}\label{chap:conclusioni}
Con la fase di test dei connettori e il raggiungimento dell'operativit\`a massima si conclude questo progetto di tesi.

Il maggiore sforzo nello studio e sviluppo di questo progetto \`e consistito nella gestione e nella ricerca di soluzioni valide per l'elusione dei controlli imposti dai servizi a discapito dell'attivit\`a di scraping.

La costanza degli aggiornamenti e l'introduzione delle nuove funzionalit\`a di opposizione all'estrazione introdotte da Facebook ed Instagram hanno fatto s\`i che il lavoro di tesi richiedesse una continua attenzione e frequenza nel testing del software, causando molteplici cambiamenti nell'approccio di sviluppo e ricercando nuove soluzioni.

Le funzioni implementate rappresentano ottimi compromessi tra operativit\`a continua ed efficienza. L'alto numero di dati raccolti e gestiti rappresenta un input ottimale per la successiva ingestione in sistemi di Big Data Analytics. 

Il mantenimento nel lungo periodo di un software stabile e facile da aggiornare rappresenta l'obiettivo finale del progetto, raggiunto grazie alle soluzioni studiate e sviluppate in questo lavoro di tesi.

Si fa presente che tutti i dati nell'ambito dello sviluppo e del test del progetto sono stati trattati ai fini di ricerca scientifica.
%\newpage

\section{Sviluppi futuri}
I principali aspetti che possono rappresentare un futuro sviluppo del progetto possono essere i seguenti:
\begin{itemize}
    \item la possibilit\`a di integrazione del software sviluppato nell'ambito dei sistemi distribuiti, come proposto al Paragrafo \ref{sistemi_distribuiti}. Questa idea implementativa rientra nelle possibilit\`a di elusione dei controlli imposti dai fornitori dei servizi social;
    \item gestione sicura degli account da impiegare tramite la memorizzazione delle credenziali crittografate. Un esempio pu\`o essere prevedere l'utilizzo di SQLite con SEE\footnote{SQLite Encryption Extension, \url{https://sqlite.org/com/see.html}}; 
    \item la creazione di un generatore automatico di account, in modo da non impiegare attivamente l'operatore nel fornire utenze manualmente;
    \item implementare ip-rotation proposta al Paragrafo \ref{ip_rotation} ed eventuali Proxy Server.
\end{itemize}

Il progetto, osservato il continuo cambiamento delle piattaforme, necessita di aggiornamento costante. Le funzioni sviluppate sono state sviluppate in modo da essere facilmente adattate ad eventuali novit\`a tecniche introdotte dai social, resta per\`o comunque l'esigenza di un continuo sviluppo, soprattutto per rafforzare l'elusione dei sistemi anti-scraping.
